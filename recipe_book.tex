\section{Aubergines Parmesanes}\label{aubergines-parmesanes}

\subsection{Ingrédients (3p)}\label{ingruxe9dients-3p}

\begin{itemize}

\item
  1 paquet d'aubergines grillées Picard (ou 3 aubergines grillées
  maison)
\item
  1 paquet de sauce à l'italienne Picard (ou sauce tomate maison)
\item
  2 boules de mozzarella
\item
  Du parmesan ou pecorino frais à râper (environ 50g)
\end{itemize}

\subsection{Notes}\label{notes}

\begin{itemize}

\item
  ``Les 3 et bien pour 3 personnes.''
\item
  ``Un peu de sel + poivre + 1 peu d'ail et origan.''
\item
  ``Fleur d'ail, pecorino, herbes origan.''
\item
  ``Fournir quelques minutes pour dorer - ça glorifie !''
\end{itemize}

\subsection{Recette}\label{recette}

\begin{enumerate}
\def\labelenumi{\arabic{enumi}.}

\item
  Préchauffez votre four à 210°C.
\item
  Faites décongeler la sauce tomate et les aubergines.
\item
  Dans un plat à gratin, alternez une couche de sauce tomate, une couche
  d'aubergines, une couche de tomate. Répétez l'opération et terminez
  par du parmesan râpé.
\item
  Enfournez pendant 40 minutes.
\item
  Servez avec une salade verte pour un repas léger ou avec une viande
  grillée. Ajoutez du basilic ou du persil frais pour décorer.
\end{enumerate}

\section{Beurre Blanc}\label{beurre-blanc}

\subsection{Ingrédients}\label{ingruxe9dients}

\begin{itemize}

\item
  Beurre (coupé en petits morceaux de 1 cm³)
\item
  Jus de citron (pour la variante au beurre au citron)
\end{itemize}

\subsection{Préparation}\label{pruxe9paration}

\begin{enumerate}
\def\labelenumi{\arabic{enumi}.}

\item
  Faites fondre un morceau de beurre dans un bain-marie à feu très doux.
\item
  Ajoutez les autres morceaux de beurre petit à petit, tout en
  mélangeant constamment.
\item
  Si le mélange commence à tourner, plongez la casserole dans de l'eau
  froide pour stabiliser la préparation.
\item
  Pour la variante au beurre au citron, ajoutez le jus de citron au
  beurre fondu en fouettant vigoureusement.
\end{enumerate}

\section{Cake aux olives et à la feta (6
pers.)}\label{cake-aux-olives-et-uxe0-la-feta-6-pers.}

\subsection{Ingrédients}\label{ingruxe9dients-1}

\begin{itemize}

\item
  200 g de farine
\item
  1 sachet de levure chimique
\item
  3 tomates
\item
  200 g de feta (coupée en dés)
\item
  100 cl de crème liquide
\item
  150 g d'olives noires dénoyautées (coupées en rondelles)
\item
  3 œufs
\item
  30 g de beurre (pour le moule)
\item
  10 cl d'huile d'olive
\item
  Sel et poivre
\end{itemize}

\subsection{Recette}\label{recette-1}

\begin{enumerate}
\def\labelenumi{\arabic{enumi}.}

\item
  Préchauffez le four à 180°C. Mélangez la farine et la levure dans un
  saladier. Ajoutez les œufs entiers. Fouettez doucement en incorporant
  l'huile puis la crème petit à petit.
\item
  Ajoutez la feta, les olives et les tomates coupées en dés à la pâte.
  Mélangez bien. Salez et poivrez.
\item
  Beurrez un moule à cake et versez-y la préparation. Enfournez pendant
  10 minutes à 180°C, puis réduisez à 150°C et poursuivez la cuisson 35
  minutes.
\item
  Laissez tiédir avant de démouler. Servez tiède ou froid, accompagné
  d'une salade.
\end{enumerate}

\section{Coquillettes aux Moules (2-3
pers.)}\label{coquillettes-aux-moules-2-3-pers.}

\subsection{Ingrédients}\label{ingruxe9dients-2}

\begin{itemize}

\item
  2-3 poignées de coquillettes
\item
  Eau salée
\item
  1 paquet de moules surgelées (de chez Picard)
\item
  1 verre de vin blanc
\item
  Jus de citron (quantité au goût)
\item
  Persil en quantité
\item
  Échalotes (quantité au goût)
\end{itemize}

\subsection{Recette}\label{recette-2}

\begin{enumerate}
\def\labelenumi{\arabic{enumi}.}

\item
  Faites cuire les coquillettes dans de l'eau salée pendant 8 à 9
  minutes (selon la marque).
\item
  Pendant ce temps, dégelez les moules dans une casserole avec le vin
  blanc, un peu de jus de citron, du persil, et des échalotes.
\item
  Mélangez les coquillettes et les moules une fois que les deux sont
  prêts, puis servez immédiatement.
\end{enumerate}

\section{Courgettes à la Brousse (4
pers.)}\label{courgettes-uxe0-la-brousse-4-pers.}

\subsection{Ingrédients}\label{ingruxe9dients-3}

\begin{itemize}

\item
  400 g de brousse
\item
  6 belles grosses courgettes
\item
  5 à 6 gousses d'ail
\item
  3 gros œufs
\item
  3 c.~à soupe de chapelure
\item
  4 tomates
\item
  Huile d'olive
\item
  Thym (selon goût), sel et poivre
\end{itemize}

\subsection{Recette}\label{recette-3}

\begin{enumerate}
\def\labelenumi{\arabic{enumi}.}

\item
  Rincez les courgettes, coupez-les en deux et blanchissez-les avec de
  l'ail pelé pendant environ 7 à 10 minutes.
\item
  Réduisez les courgettes en purée avec la brousse et mélangez avec les
  jaunes d'œufs et le thym.
\item
  Farcissez les courgettes, décorez de tomates et saupoudrez de
  chapelure, huile d'olive, sel et poivre.
\item
  Faites cuire au four à 220°C pendant 20 minutes. Gratinez si
  nécessaire.
\end{enumerate}

\section{Croûte aux Champignons (4
pers.)}\label{crouxfbte-aux-champignons-4-pers.}

\subsection{Ingrédients}\label{ingruxe9dients-4}

\subsubsection{Champignons}\label{champignons}

\begin{itemize}

\item
  1 échalote
\item
  1 gousse d'ail
\item
  30 g de beurre
\item
  1 paquet de morilles (surgelées ou fraîches)
\item
  500 g de champignons de Paris
\item
  1 paquet de cèpes
\item
  5 cl de cognac
\item
  20 cl de vin jaune ou savagnin
\end{itemize}

\subsubsection{Sauce}\label{sauce}

\begin{itemize}

\item
  50 g de beurre
\item
  20 cl de lait
\item
  1 cube de bouillon de volaille
\item
  250 g de crème épaisse
\item
  Sel, poivre, piment d'Espelette
\end{itemize}

\subsubsection{Accompagnement}\label{accompagnement}

\begin{itemize}

\item
  Rösties Picard (ou autre garniture)
\item
  Tranches de pain grillé
\end{itemize}

\subsection{Préparation}\label{pruxe9paration-1}

\begin{enumerate}
\def\labelenumi{\arabic{enumi}.}

\item
  Faites revenir l'échalote et l'ail dans le beurre, puis ajoutez les
  morilles et les champignons de Paris. Laissez cuire à feu doux.
\item
  Égouttez les champignons et réservez leur jus. Faites flamber les
  champignons avec le cognac.
\item
  Dans une casserole, préparez la sauce en faisant fondre le beurre et
  en ajoutant le lait, le cube de bouillon, et la crème épaisse.
  Assaisonnez avec le sel, le poivre, et le piment d'Espelette.
\item
  Ajoutez les champignons et mélangez bien.
\item
  Servez les champignons sur des tranches de pain grillé ou des rôtis et
  nappez de sauce.
\end{enumerate}

\section{Effilochée de Raie Marinée au Fenouil et Poivrons (4
pers.)}\label{effilochuxe9e-de-raie-marinuxe9e-au-fenouil-et-poivrons-4-pers.}

\subsection{Ingrédients}\label{ingruxe9dients-5}

\begin{itemize}

\item
  1 kg d'ailes de raie pelées
\item
  300 g de poivrons grillés
\item
  450 g de cœurs de fenouil
\item
  2 cuil. à soupe de basilic ciselé
\item
  1 cuil. à soupe d'échalote coupée
\item
  4 tablettes de court-bouillon
\item
  2 citrons
\item
  4 cuil. à soupe d'huile d'olive
\item
  Sel et poivre
\end{itemize}

\subsection{Recette}\label{recette-4}

\begin{enumerate}
\def\labelenumi{\arabic{enumi}.}

\item
  Préparez un court-bouillon avec les tablettes, l'huile d'olive, et le
  jus d'un citron. Portez à ébullition.
\item
  Déposez les morceaux de raie dans le court-bouillon, laissez cuire 12
  à 15 minutes, puis égouttez.
\item
  Dans un bol, mélangez le fenouil, les poivrons, et les échalotes.
  Ajoutez du sel, du poivre, et le basilic.
\item
  Servez en accompagnant de tranches de citron.
\end{enumerate}

\section{Filet Mignon (2 pers.)}\label{filet-mignon-2-pers.}

\subsection{Ingrédients}\label{ingruxe9dients-6}

\begin{itemize}

\item
  Filets mignons de porc (2pers. par filet env)
\item
  Huile d'olive
\item
  Oignons blancs
\item
  Beurre
\item
  Romarin ou sauge
\item
  Crème fraîche
\item
  Sel et poivre
\end{itemize}

\subsection{Préparation}\label{pruxe9paration-2}

\begin{enumerate}
\def\labelenumi{\arabic{enumi}.}

\item
  Dans une cocotte, faites dorer les filets mignons dans un peu d'huile
  d'olive. Retirez et réservez.
\item
  Dans la même cocotte, faites fondre les oignons blancs avec un peu de
  beurre.
\item
  Ajoutez le romarin et/ou la sauge hachés finement, puis versez une
  petite quantité de crème fraîche.
\item
  Remettez les filets mignons dans la cocotte, salez, poivrez, et
  ajoutez de la crème pour couvrir légèrement.
\item
  Couvrez la cocotte et faites cuire à feu doux pendant environ 45
  minutes. Vérifiez la cuisson avec une pique.
\item
  Tranchez le filet mignon et nappez-le avec la crème avant de servir.
\end{enumerate}

\section{Pâtes aux Champignons et Miso (2
pers.)}\label{puxe2tes-aux-champignons-et-miso-2-pers.}

\subsection{Ingrédients}\label{ingruxe9dients-7}

\begin{itemize}

\item
  200 g de champignons de Paris
\item
  1 petit oignon
\item
  1 gousse d'ail
\item
  15 cl de crème liquide vegan
\item
  1 c.~à c.~de miso blanc
\item
  1 c.~à s. de sauce soja
\item
  2 c.~à c.~de saké ou vin blanc de cuisson
\item
  Pâtes pour deux personnes
\end{itemize}

\subsection{Recette}\label{recette-5}

\begin{enumerate}
\def\labelenumi{\arabic{enumi}.}

\item
  Faites cuire les pâtes.
\item
  Coupez finement l'oignon et l'ail, émincez les champignons.
\item
  Dans un bol, mélangez le miso, la sauce soja, et la crème.
\item
  Dans une poêle, faites revenir l'oignon, puis ajoutez l'ail et les
  champignons.
\item
  Ajoutez le mélange miso-crème et laissez mijoter quelques minutes.
\item
  Ajoutez les pâtes et servez.
\end{enumerate}

\section{Piccata de veau à la sauge (4
pers.)}\label{piccata-de-veau-uxe0-la-sauge-4-pers.}

\subsection{Ingrédients}\label{ingruxe9dients-8}

\begin{itemize}

\item
  8 petites escalopes de veau (125 g chacune)
\item
  8 tranches fines de poitrine fumée
\item
  8 feuilles de sauge (par escalope)
\item
  Farine
\item
  125 g de beurre
\item
  20 cl de marsala sec
\item
  Sel et poivre
\item
  Feuilles de sauge pour garnir
\end{itemize}

\subsection{Notes}\label{notes-1}

\begin{itemize}

\item
  Attention à acheter le marsala NATURE
\end{itemize}

\subsection{Recette}\label{recette-6}

\begin{enumerate}
\def\labelenumi{\arabic{enumi}.}

\item
  Amincir les escalopes en les plaçant entre deux couches de Cellophane
  et aplatir avec une casserole (ou demander au boucher).
\item
  Parez les escalopes en éliminant les morceaux de cartilage et les
  petites peaux. Posez une tranche de poitrine et une feuille de sauge
  sur chaque escalope. Maintenez en place avec un cure-dent.
\item
  Salez légèrement et passez les escalopes dans la farine de chaque
  côté.
\item
  Faites chauffer la moitié du beurre dans une poêle et faites dorer les
  escalopes pendant environ 2 minutes de chaque côté. Attention à ne pas
  faire trop cuire les escalopes. Réservez au chaud.
\item
  Versez le marsala dans la poêle, déglacez, et ajoutez le reste de
  beurre. Portez à ébullition et nappez les escalopes de la sauce.
  Garnissez de feuilles de sauge fraîche et servez aussitôt.
\end{enumerate}

\section{Polenta au Romarin et Parmesan
(5-6pers.)}\label{polenta-au-romarin-et-parmesan-5-6pers.}

\subsection{Ingrédients}\label{ingruxe9dients-9}

\begin{itemize}

\item
  1 volume de polenta (500g par paquet)
\item
  2,5 volumes d'eau (ou moitié eau, moitié lait): verifier le paquet
  pour proportions
\item
  1 cube de bouillon de légumes
\item
  Romarin (quantité au goût)
\item
  16.5cl Crème fraîche (un demi paquet)
\item
  100g env. Parmesan râpé
\end{itemize}

\subsection{Notes}\label{notes-2}

\begin{itemize}

\item
  Se congèle très bien. Mais doit être réchauffé a la casserole en le
  cassant en plus petit morceaux car au micro-ondes cela se dissocie. On
  peut ajouter une cuillère de crème pendant que l'on réchauffe.
\item
  Peut aussi être poêlé (comme le font les italiens)
\end{itemize}

\subsection{Préparation}\label{pruxe9paration-3}

\begin{enumerate}
\def\labelenumi{\arabic{enumi}.}

\item
  Faites bouillir l'eau (ou le mélange d'eau et de lait) avec le
  bouillon de légumes.
\item
  Ajoutez le romarin, puis versez la polenta en pluie (assez haut), tout
  en remuant constamment pour éviter les grumeaux.
\item
  Continuez à mélanger pendant 3 à 5 minutes, puis ajoutez la crème et
  le parmesan.
\item
  Remuez jusqu'à ce que la polenta soit crémeuse mais encore légèrement
  tendre.
\item
  Ajouter du romarin frais par dessus.
\item
  Servez aussitôt.
\end{enumerate}

\section{Poulet à l'Estragon (6-8
pers.)}\label{poulet-uxe0-lestragon-6-8-pers.}

\subsection{Ingrédients}\label{ingruxe9dients-10}

\begin{itemize}

\item
  8 morceaux de poulet (cuisses et blancs)
\item
  3 contenants de crème liquide epaisse légère (33cl x3)
\item
  1 grosse botte d'estragon frais
\item
  2-3 cuillères à soupe d'huile (pour la cuisson)
\item
  1 poignée d'oignons blancs
\item
  Sel et poivre
\end{itemize}

\section{Notes}\label{notes-3}

\begin{itemize}

\item
  Se marie bien avec du riz \#\# Recette
\end{itemize}

\begin{enumerate}
\def\labelenumi{\arabic{enumi}.}

\item
  Dans une cocotte, faites chauffer l'huile et faites rôtir les morceaux
  de poulet jusqu'à ce que la peau soit bien dorée. Retirez le poulet de
  la cocotte et réservez.
\item
  Ajoutez les oignons dans la cocotte et faites-les fondre doucement
  dans le beurre. Ajoutez un peu d'estragon, du sel, du poivre, et une
  portion de crème. Mélangez bien.
\item
  Remettez les morceaux de poulet dans la cocotte et recouvrez-les avec
  le reste de la crème et des herbes. Faites cuire à feu très doux
  (environ 4-5 sur 14 niveaux) pendant 1 heure à 1h15, en remuant de
  temps en temps.
\item
  Vérifiez la cuisson, puis retirez les morceaux de poulet. Ajoutez de
  l'estragon frais dans la sauce et mélangez bien.
\item
  Servez le poulet dans un plat et arrosez-le de la sauce crémeuse à
  l'estragon.
\end{enumerate}

\section{Raie au Beurre Noisette (4
pers.)}\label{raie-au-beurre-noisette-4-pers.}

\subsection{Ingrédients}\label{ingruxe9dients-11}

\begin{itemize}

\item
  4 morceaux d'aile de raie de 250 g
\item
  2 cuillères à soupe de vinaigre
\item
  200 g de beurre
\item
  1 citron
\item
  2 cuillères à soupe de câpres
\item
  Sel et poivre
\item
  Thym, persil et laurier
\end{itemize}

\subsection{Recette}\label{recette-7}

\begin{enumerate}
\def\labelenumi{\arabic{enumi}.}

\item
  Dans une grande casserole d'eau froide, disposez les morceaux de raie,
  le thym, le persil, et le laurier avec 1 cuillère à soupe de vinaigre.
  Salez et portez à ébullition. Réduisez le feu et laissez cuire environ
  10 minutes.
\item
  Dans une poêle chaude, faites fondre le beurre avec 1 cuillère à soupe
  de vinaigre. Ajoutez du persil frais et les câpres, puis assaisonnez
  avec du sel et du poivre.
\item
  Disposez les ailes de raie égouttées dans des assiettes, puis
  nappez-les de beurre noisette. Ajoutez une rondelle de citron pour
  chaque assiette et servez avec du riz ou des pommes de terre vapeur.
\end{enumerate}

\section{Raviolis au Beurre de Sauge}\label{raviolis-au-beurre-de-sauge}

\subsection{Ingrédients}\label{ingruxe9dients-12}

\begin{itemize}

\item
  Raviolis au fromage (deux douzaines)
\item
  Feuilles de sauge
\item
  Huile
\item
  Beurre
\item
  Sel et poivre
\end{itemize}

\subsection{Préparation}\label{pruxe9paration-4}

\begin{enumerate}
\def\labelenumi{\arabic{enumi}.}

\item
  Dans une casserole, faites chauffer un mélange d'huile et de beurre.
\item
  Ajoutez les feuilles de sauge et faites-les griller doucement jusqu'à
  ce qu'elles brunissent légèrement. La couleur de la feuille guide la
  cuisson. Ne les laissez pas cramer !
\item
  Assaisonnez avec du sel et du poivre. Servez les raviolis nappés du
  beurre de sauge.
\end{enumerate}

\section{Salade d'artichauts aux fèves et au citron (4
pers.)}\label{salade-dartichauts-aux-fuxe8ves-et-au-citron-4-pers.}

\subsection{Ingrédients}\label{ingruxe9dients-13}

\begin{itemize}

\item
  300 g de cœurs d'artichauts
\item
  250 g de fèves
\item
  1 grappe de tomates cerises
\item
  1 citron bio
\item
  2 gousses d'ail
\item
  75 g de roquette
\item
  3 cuil. à soupe d'huile d'olive
\item
  1 cuil. à soupe de vinaigre balsamique
\item
  Sel, poivre, thym
\end{itemize}

\subsection{Recette}\label{recette-8}

\begin{enumerate}
\def\labelenumi{\arabic{enumi}.}

\item
  Préchauffez le four à 180°C. Étalez les feuilles de sauge dans un plat
  huilé.
\item
  Ajoutez les tomates cerises, arrosez d'huile, salez, et poivrez.
  Faites cuire 10 minutes.
\item
  Faites cuire les fèves dans de l'eau salée, égouttez, et plongez-les
  dans de l'eau froide.
\item
  Mélangez les cœurs d'artichauts, les fèves, la roquette, et les
  tomates cerises rôties. Assaisonnez de vinaigre et servez.
\end{enumerate}

\section{Salade d'avocats et pamplemousses aux crevettes (4
pers.)}\label{salade-davocats-et-pamplemousses-aux-crevettes-4-pers.}

\subsection{Ingrédients}\label{ingruxe9dients-14}

\begin{itemize}

\item
  350 g de crevettes roses
\item
  1 pamplemousse
\item
  2 avocats
\item
  2 citrons
\item
  3 cuil. à soupe d'huile d'olive
\item
  Brins de ciboulette
\item
  Tabasco
\end{itemize}

\subsection{Recette}\label{recette-9}

\begin{enumerate}
\def\labelenumi{\arabic{enumi}.}

\item
  Décortiquez les crevettes et faites-les mariner dans une assiette avec
  le jus d'un citron, l'huile d'olive, et quelques gouttes de Tabasco.
\item
  Pelez le pamplemousse et les avocats, puis coupez-les en lamelles.
  Arrosez-les du jus de l'autre citron.
\item
  Mélangez les lamelles d'avocat, de pamplemousse, et les crevettes.
  Parsemez de ciboulette ciselée et servez très frais.
\end{enumerate}

\section{Salade de Roquette à l'Artichaut, Jambon Cru et Parmesan (4
pers.)}\label{salade-de-roquette-uxe0-lartichaut-jambon-cru-et-parmesan-4-pers.}

\subsection{Ingrédients}\label{ingruxe9dients-15}

\begin{itemize}

\item
  1 c.~à soupe de jus de citron
\item
  3 c.~à soupe d'huile d'olive
\item
  Sel, poivre
\item
  1 gros artichaut
\item
  75 g de jambon cru
\item
  60 g de parmesan
\item
  100 g de feuilles de roquette
\end{itemize}

\subsection{Recette}\label{recette-10}

\begin{enumerate}
\def\labelenumi{\arabic{enumi}.}

\item
  Mélangez le jus de citron, l'huile d'olive, le sel, et le poivre dans
  un bol.
\item
  Préparez l'artichaut, coupez-le en lamelles, et ajoutez-le à la
  vinaigrette.
\item
  Ajoutez le jambon cru et le parmesan en copeaux. Mélangez le tout avec
  la roquette et servez.
\end{enumerate}

\section{Salade Verte Sauce}\label{salade-verte-sauce}

\subsection{Ingrédients}\label{ingruxe9dients-16}

\begin{itemize}

\item
  3 c.~à soupe d'huile d'olive
\item
  3 dash de sauce Maggi
\item
  1 c.~à soupe Vinaigre maison (quantité au goût)
\item
  1/2 c.~à soupe d'échalote hachée
\item
  Ciboulette (quantité au goût)
\item
  Sel et poivre
\end{itemize}

\subsection{Notes}\label{notes-4}

\begin{itemize}

\item
  pour les autres salades préférer d'autres vinaigre.
\item
  pour certaines salades ajouter de l'ail
\item
  si pas de vinaigre maison: construire une c.~à soupe avec 1 bouchon
  vinaigre de cidre, 1 bouchon vinaigre balsamique, 1 bouchon xérès.
\item
  Les herbes congelés de picard sont très pratique. \#\# Préparation
\end{itemize}

\begin{enumerate}
\def\labelenumi{\arabic{enumi}.}

\item
  Mélangez l'huile d'olive, la sauce Maggi, les vinaigres, les
  échalotes, et la ciboulette.
\item
  Assaisonnez avec du sel et du poivre. Ajustez selon votre goût.
\end{enumerate}

\section{Soupe aux Coquillettes et
Parmesan}\label{soupe-aux-coquillettes-et-parmesan}

\subsection{Ingrédients}\label{ingruxe9dients-17}

\begin{itemize}

\item
  2-3 poignées de coquillettes
\item
  1 à 1,5 litre d'eau
\item
  1 cube Bouillon de poulet (quantité au goût)
\item
  2 cubes Bouillon de légumes (quantité au goût)
\item
  2 œufs
\item
  2-3 cuillères à soupe de parmesan
\item
  2-3 cuillères de persil
\item
  Sel et poivre
\end{itemize}

\subsection{Recette}\label{recette-11}

\begin{enumerate}
\def\labelenumi{\arabic{enumi}.}

\item
  Portez l'eau à ébullition avec le bouillon de poulet et le bouillon de
  légumes.
\item
  Ajoutez les coquillettes et faites cuire environ 10 minutes, jusqu'à
  ce qu'elles soient tendres.
\item
  Pendant ce temps, cassez les œufs dans une soupière, ajoutez le
  parmesan, le persil, le sel et le poivre, puis mélangez.
\item
  Une fois les coquillettes cuites, versez-les rapidement dans la
  soupière en mélangeant énergiquement pour que les œufs se coagulent
  sans cuire.
\item
  Servez immédiatement.
\end{enumerate}

\section{Steak Rôti et Sauce au
Jus}\label{steak-ruxf4ti-et-sauce-au-jus}

\subsection{Ingrédients}\label{ingruxe9dients-18}

\begin{itemize}

\item
  Steaks
\item
  Huile
\item
  Eau
\item
  Beurre
\item
  Sel et poivre
\end{itemize}

\subsection{Préparation}\label{pruxe9paration-5}

\begin{enumerate}
\def\labelenumi{\arabic{enumi}.}

\item
  Chauffez une poêle avec un peu d'huile. Faites cuire les steaks à feu
  très chaud, en les retournant pour bien les rôtir de chaque côté.
\item
  Enveloppez les steaks dans du papier aluminium et laissez reposer
  pendant 10 minutes.
\item
  Déglacez la poêle avec un peu d'eau pour récupérer les sucs, puis
  ajoutez un morceau de beurre. Mélangez bien, salez et poivrez.
\item
  Ouvrez les steaks sur la poêle pour laisser le jus se mélanger à la
  sauce. Servez aussitôt. Crumble salé et sucré ! polenta haddock
  lentille au beurre blanc bearnaise -\textgreater{} louis dos de
  cabillaud à la sauce vierge tarte au lime croute au champi salade de
  pecheur minestrone \# Tarte chèvre et tomates (6 pers.)
\end{enumerate}

\subsection{Ingrédients}\label{ingruxe9dients-19}

\begin{itemize}

\item
  500 g de tomates cerises
\item
  25 g de comté râpé
\item
  250 g de farine
\item
  125 g de beurre
\item
  1 œuf
\item
  1 cuillère à soupe de lait ou d'eau
\item
  Sel, poivre, piment, sucre roux, graines de sésame
\end{itemize}

\subsection{Recette}\label{recette-12}

\begin{enumerate}
\def\labelenumi{\arabic{enumi}.}

\item
  Préchauffez le four à 180°C. Mélangez les farines avec le beurre, le
  sel, le poivre, et le piment.
\item
  Ajoutez le lait ou l'eau pour former une pâte, puis étalez-la sur une
  plaque.
\item
  Répartissez les tomates cerises sur la pâte. Saupoudrez de comté,
  sucre roux, et graines de sésame.
\item
  Faites cuire au four environ 45 minutes. Servez avec une salade de
  roquette.
\end{enumerate}

\section{Boîte chaude de Vacherin Mont-d'Or (4
pers.)}\label{bouxeete-chaude-de-vacherin-mont-dor-4-pers.}

\subsection{Ingrédients}\label{ingruxe9dients-20}

\begin{itemize}

\item
  1 Vacherin Mont-d'Or de 500 g
\item
  5 cl de vin blanc du Jura
\item
  1 gousse d'ail
\end{itemize}

\subsection{Recette}\label{recette-13}

\begin{enumerate}
\def\labelenumi{\arabic{enumi}.}

\item
  Préchauffez le four à 220°C. Ôtez le couvercle du Vacherin Mont-d'Or
  et replacez-le dans sa boîte.
\item
  Faites un trou au centre du fromage, versez le vin blanc, ajoutez
  l'ail coupé en lamelles.
\item
  Enfournez pendant 30 minutes et servez avec des pommes de terre.
\end{enumerate}

\section{Vinaigre Maison}\label{vinaigre-maison}

\subsection{Ingrédients}\label{ingruxe9dients-21}

\begin{itemize}

\item
  1/2 vinaigre de vin
\item
  1/4 vinaigre de xérès
\item
  1/4 vinaigre balsamique
\item
  Brins d'estragon (facultatif)
\end{itemize}

\subsection{Préparation}\label{pruxe9paration-6}

\begin{enumerate}
\def\labelenumi{\arabic{enumi}.}

\item
  Mélangez le vinaigre de vin, le vinaigre de xérès, et le vinaigre
  balsamique dans une bouteille.
\item
  Ajoutez un brin d'estragon pour aromatiser, si désiré.
\end{enumerate}
